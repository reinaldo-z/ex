\newcommand{\seccion}{SECUNDARIA INCORPORADA A LA SEG }
\newcommand{\descripcion}{Examen Parcial, Cuarto Bimestre}
\newcommand{\grado}{Primero de secundaria}
\newcommand{\ciclo}{Ciclo escolar: 2015--2016}
\newcommand{\papel}{legalpaper} %letter, legalpaper ...
\newcommand{\fecha}{22 de abril de 2016}
% \author{Ing. Arturo Canedo \\ M. en C. Reinaldo Zapata}
\author{M. en C. Reinaldo Zapata}

\documentclass[11pt]{article}
\usepackage[\papel]{geometry}

\title{\flushleft \seccion \\ \descripcion \\  \grado \\ \ciclo}

\newcommand\BackgroundLogo{
\put(162,365){
\parbox[b][\paperheight]{\paperwidth}{%
\vfill
\centering
\includegraphics[width=5cm,height=2.5cm,keepaspectratio]{/Users/reinaldo/Documents/clases/jassa/logo}%
\vfill
}}}



% \title{\seccion \\ \descripcion \\  \grado \\ \ciclo}

% \newcommand\BackgroundLogo{
% \put(162,435){
% \parbox[b][\paperheight]{\paperwidth}{%
% \vfill
% \centering
% \includegraphics[width=5cm,height=2.5cm,keepaspectratio]{/Users/reinaldo/Documents/clases/jassa/logo}%
% \vfill
% }}}

\hyphenation{con-ti-nua-ción}

\usepackage{enumitem}
\usepackage[T1]{fontenc} %fuentes
\usepackage{lmodern} %fuente mejorada
\usepackage[spanish]{babel}
\decimalpoint
\usepackage{fullpage}
\usepackage{multicol}
\usepackage{graphicx}
\usepackage{eso-pic}
\usepackage{multirow}
\usepackage{subfigure}
\usepackage{tikz}
\usetikzlibrary{shapes.geometric}



%Modificación del formato de las ecuaciones y el numerado de las mismas
\usepackage[leqno,fleqn]{amsmath}
\makeatletter
  \def\tagform@#1{\maketag@@@{#1\@@italiccorr}}
\makeatother
\renewcommand{\theequation}{\fbox{\textbf{\arabic{equation}}}}


\begin{document}
\AddToShipoutPicture*{\BackgroundLogo}
\ClearShipoutPicture
\date{\fecha}
\maketitle
% \thispagestyle{empty}



Nombre del alumno:\,\line(1,0){244}\,.\hspace*{.2cm} \hfill Aciertos:
$\dfrac{\qquad \qquad}{50}$.

\vspace{3mm}
\indent Primero de secundaria, grupo:\,\line(1,0){35}\,. No. de
lista:\,\line(1,0){35}\,.

\vspace{5mm}

El prop\'osito de todo examen es poner a prueba los conocimientos de cada alumno
para calificar as\'i su desempe\~no y aprendizaje. Contesta correctamente, en
cada secci\'on, tantos reactivos como te sea posible. Todas y cada una de las
operaciones deber\'as escribirlas en esta hoja donde se imprimi\'o el examen y
los resultados finales deber\'an estar escritos con tinta.

\section{N\'umeros con signo}

Escribe los cuatro pasos para hacer sumas y restas de n\'umeros con signo.
\hfill \textbf{(8 aciertos)}

\begin{enumerate}
    \item \ Sumar todos los n\'umeros positivos \\
    \item \ Sumar todos los n\'umeros negativos \\
    \item \ Restar las cantidades anteriores \\
    \item \ Elegir el signo, + o -, del que hay sido mayor. \\
\end{enumerate}

Resuelve las siguientes operaciones de n\'umeros con signo. \hfill \textbf{(10
aciertos)}

\begin{multicols}{2}

\begin{equation}
-3 +5 -6 +8 -6 = -2
\end{equation}

\vspace{3cm}
\begin{equation}
-20 +10 -15 -9 = -34
\end{equation}

\vspace{3cm}
\begin{equation}
-4.16 +2.5 -3.15 + 0.975 = -3.835
\end{equation}

\vspace{3cm}
\begin{equation}
- \frac{2}{5} + \frac{2}{3} + \frac{1}{15} - \frac{1}{2} = - \frac{1}{6}
\end{equation}

\vspace{3cm}

\end{multicols}

\newpage

\setcounter{equation}{0}
\section{Ecuaciones}

Resuelve las ecuaciones que se plantean a continuaci\'on. \hfill \textbf{(4 aciertos)}

\begin{multicols}{2}

\begin{align}
3x + 6 &= 12 \\
3x &= 6 \nonumber \\
x &= 2 \nonumber
\end{align}
\vfill 
\vspace{3cm}
\begin{align}
9 + \frac{2}{3} x &= -\frac{2}{5} \\
\frac{2}{3}x &= - \frac{2}{5} - \frac{9}{1} \nonumber \\
x &= - \frac{47}{5} \div \frac{2}{3} \nonumber \\
x &= - \frac{141}{10}  = - 14 \frac{1}{10} \nonumber
\end{align}


\end{multicols}

Resuelve planteando la ecuaci\'on correspondiente: si al triple de un n\'umero
se le agregan 10 da como resultado 20. ?`Cu\'al es ese n\'umero? \hfill
\textbf{(4 aciertos)}

\begin{align}
3x +10 &= 20  \nonumber \\
x= \frac{10}{3} \nonumber   
\end{align}

\vspace{0.5cm}

\section{Probabilidad y estad\'istica}
En m\'exico hay cerca de 125 millones de habitantes. Acorde a estudios hechos
por la Secretar\'ia de Econom\'ia acerca de las clases sociales en M\'exico se
tiene que 7.5 millones de ellos pertenece a la clase \emph{ata}, 42.5 millones a
la clase \emph{media} y el resto a la clase \emph{baja}. Usando la
circunferencia que se muestra a continuaci\'on, expresa los porcentajes de los
datos haciendo una gr\'afica circular. Recuerda colocar los porcentajes sobre la
gr\'afica. \hfill \textbf{(10 aciertos)}

\vspace{3mm}
\begin{minipage}{0.75\linewidth}
\begin{tikzpicture}[scale=0.9]
    \draw[line width=0.2mm, black, fill=blue!50] (0,0) circle (3.9cm);
    \fill[fill=red!50] (0,0) -- (0cm,3.9cm) arc (90:68.4:3.9cm);
    \fill[fill=green!50] (0,0) -- (1.4356857555cm,3.626128295cm) arc (68.4:-54:3.9cm);

    \draw[line width=0.2mm, black,] (0,0) circle (3.9cm);
    \draw [line width=0.2mm, black ] (0,0) -- (0,3.9) node [right] {};;
    \draw [line width=0.5mm, black ] (0,-0.2) -- (0,0.2) node [right] {};;
    \draw [line width=0.5mm, black ] (-0.2,0) -- (0.2,0) node [right] {};;
    \draw[line width=0.2mm, black, fill=red!50] (4.5,1.5)  circle (3mm) node [right] {\Large \hspace{6mm}Alta};
    \draw[line width=0.2mm, black, fill=green!50] (4.5,0.0)  circle (3mm) node [right] {\Large \hspace{6mm}Media};
    \draw[line width=0.2mm, black, fill=blue!50] (4.5,-1.5) circle (3mm) node [right] {\Large \hspace{6mm}Baja};

    \draw [] (0,2.8) node [right] {\Large 6\%};
    \draw [] (2.8,0) node [center] {\Large 34\%};
    \draw [] (-2.42,-1.4) node [center] {\Large 60\%};

\end{tikzpicture}  
\end{minipage}\hspace{-1.5cm}
\begin{minipage}{0.3\linewidth}
Alta

125 millones \hfill 100\%

7.5 millones \hfill x = 6\% \\

Media

125 millones \hfill 100\%

42.5 millones \hfill x = 34\% \\

Baja

125 millones \hfill 100\%

7.5 millones \hfill x = 60\% \\
\end{minipage}

\vspace{3mm}
El Instituto Nacional de Estad\'iistica y Geograf\'ia (INEGI) obtuvo como
resultado del censo en el a\~no 2010 que el 48.5\% de los mexicanos son hombres
mientras que el restante son mujeres. Acorde a estos datos etiqueta de forma
correcta la gr\'afica que se muestra a continuaci\'on colocando en los
c\'irculos peque\~nos las palabras \emph{hombres} y \emph{mujeres} y
poniendo adem\'as los porcentajes correspondientes sobre la gr\'afica. \hspace{10cm} \textbf{(4 aciertos)}


\vspace{3mm}
\begin{tikzpicture}[scale=0.63]
  \fill[fill=gray!80]
    (0,0) -- (0cm,3cm) arc (90:-84:3cm);
    \draw[line width=0.2mm, black] (0,0) circle (3cm);

    \draw[line width=0.2mm, black, fill=gray!80] (4,1.5) circle (3mm) node [right] {\Large \hspace{6mm}Hombres};
    \draw[line width=0.2mm, black] (4,-1.5) circle (3mm) node [right] {\Large \hspace{6mm}Mujeres};

    \draw [] (0,0) node [right] {\hspace{3mm}48.5\%};
    \draw [] (0,0) node [left] {51.5\%\hspace{3mm}};
\end{tikzpicture}  

\section{Geometr\'ia}
Haciendo uso de la circunferencia que se muestra a continuaci\'on haz el trazo
de un pent\'agono regular inscrito. Traza su apotema. 

Asumiendo que su longitud lateral
es $\ell=5.9$\,cm y la de su apotema $a=4$\,cm, calcula su per\'imetro y su
\'area. \hfill \textbf{(5 aciertos)}

\vspace{3mm}

\begin{minipage}{0.75\linewidth}
\begin{tikzpicture}[scale=0.85]
\draw[line width=0.2mm, black] (0,0) circle (5cm);
\node [draw, thick, minimum size=8.45cm, regular polygon, regular polygon sides=5] at (0,0) { }; 
\draw [line width=0.4mm, red ] (0,0) -- (0,-4.01) node [right] {};
\draw [line width=0.2mm, black ] (0,-0.2) -- (0,0.2) node [right] {};
\draw [line width=0.2mm, black ] (-0.2,0) -- (0.2,0) node [right] {};
\end{tikzpicture}  
\end{minipage}\hspace{-2.5cm}
\begin{minipage}{0.3\linewidth}
\begin{align*}
P &= n \ell \\
  &= (5)(5.9\text{\,cm})  \\
  &= 29.5\text{\,cm} \\
\\
A &= \frac{Pa}{2} \\
  &= \frac{(29.5\text{\,cm})(4\text{\,cm})}{2} \\
  &= 59\text{\,cm}^{2}
\end{align*}
\end{minipage}

\vspace{3mm}
Escribe el nombre de las siguientes figuras sin faltas de ortograf\'ia. \hfill
\textbf{(3 aciertos)}

\vspace{3mm}
\begin{center}
\begin{tikzpicture}
\node [draw, thick, minimum size=4cm, regular polygon, regular polygon sides=7]
at (0,0) { };
\node [draw, thick, minimum size=4cm, regular polygon, regular polygon sides=5]
at (5,0) { };
\node [draw, thick, minimum size=4cm, regular polygon, regular polygon sides=10]
at (10,0) { };

\draw [line width=0.2mm, black ] (-2,-3) -- (2,-3) node [right] {};
\draw [] (0,-2.7) node [north] {\Large Hept\'agono};

\draw [line width=0.2mm, black ] (3,-3) -- (7,-3) node [right] {};
\draw [] (5,-2.7) node [north] {\Large Pent\'agono};

\draw [line width=0.2mm, black ] (8,-3) -- (12,-3) node [right] {};
\draw [] (10,-2.7) node [north] {\Large Dec\'agono};
\end{tikzpicture}
\end{center}


\setcounter{equation}{0}
\vspace{5mm}
\section{Operaciones b\'asicas}
Resuelve las siguientes operaciones. \hfill \textbf{(2 aciertos)}

\begin{multicols}{2}

\begin{equation}
(5.2)(3.1416) = 16.33632
\end{equation}

\begin{equation}
7.35 \div 2.4 = 3.06
\end{equation}


\end{multicols}



\end{document}






