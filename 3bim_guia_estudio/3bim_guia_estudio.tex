\newcommand{\seccion}{SECUNDARIA INCORPORADA A LA SEG }
\newcommand{\descripcion}{Repaso Bimestral, Tercer Bimestre }
\newcommand{\grado}{Primero de secundaria}
\newcommand{\ciclo}{Ciclo escolar: 2015--2016}
\newcommand{\papel}{letterpaper} %letterpaper, legalpaper ...
\newcommand{\fecha}{24 de febrero de 2015}

\newcommand{\oh}{\omit\hfill}
\newcommand{\tb}{\textbf}

\author{Ing. Arturo Canedo \\ M. en C. Reinaldo Zapata}

\documentclass[11pt]{article}
\usepackage[\papel]{geometry}

\title{\flushleft \seccion \\ \descripcion \\  \grado \\ \ciclo}

\newcommand\BackgroundLogo{
\put(160,265){
\parbox[b][\paperheight]{\paperwidth}{%
\vfill
\centering
\includegraphics[width=5cm,height=2.5cm,keepaspectratio]{/Users/reinaldo/Documents/clases/jassa/logo}%
\vfill
}}}

% \hyphenation{con-ti-nua-ci\'on}

\usepackage{enumitem}
\usepackage[T1]{fontenc} %fuentes
\usepackage{lmodern} %fuente mejorada
\usepackage[spanish]{babel}
\decimalpoint
\usepackage{fullpage}
\usepackage{multicol}
\usepackage{graphicx}
\usepackage{eso-pic}
\usepackage{multirow}
\usepackage{subfigure}
\usepackage{tikz}
\usepackage{hyperref} 
\usepackage{color}
\usepackage{multicol}
\usepackage{tikz}
\usetikzlibrary{shapes.geometric}



\usepackage[leqno,fleqn]{amsmath}
\makeatletter
  \def\tagform@#1{\maketag@@@{#1\@@italiccorr}}
\makeatother
\renewcommand{\theequation}{\fbox{\textbf{\arabic{equation}}}}


\begin{document}
\AddToShipoutPicture*{\BackgroundLogo}
\ClearShipoutPicture
\date{\fecha}
\maketitle

Nombre del alumno:\,\line(1,0){244}\,.\hspace*{.2cm} No. de lista:\,\line(1,0){35}\,.

Primero de secundaria, grupo:\,\line(1,0){30}\,.

\vspace{5mm}

El prop\'osito de esta gu\'ia de estudio es proporcionar al alumno un repaso de
los temas que ser\'an vistos en el examen bimestral. Contesta de forma correcta
cada uno de los reactivos que se muestran a continuaci\'on. Escribe las
operaciones y respuestas en esta hoja de estudio de forma ordenada.

\section{Fracciones y decimales}

Convierte cada fracci\'on a decimal y viceversa y despu\'es selecciona la
respuesta correcta. Recuerda simplificar tus resultados si es posible.

\begin{multicols}{2}

\begin{equation} 
\frac{5}{4} =  \oh \tb{a)}\, 1.25 \quad \tb{b)}\, 1.15 \quad
\end{equation}

\vspace{1cm}

\begin{equation} 
\frac{15}{24} =  \oh \tb{a)}\, 0.524  \quad \tb{b)}\, 0.625 \quad
\end{equation}

\vspace{1cm}

\begin{equation} 
4\frac{3}{12} =  \oh \tb{a)}\, 4.25 \quad \tb{b)}\, 0.25 \quad
\end{equation}

\vspace{1cm}

\begin{equation} 
0.23 =  \oh \tb{a)}\, \frac{23}{100}  \quad \tb{b)}\, \frac{23}{1000} \quad
\end{equation}

\vspace{1cm}

\begin{equation} 
9.275 =  \oh \tb{a)}\, 9 \frac{11}{40} \quad \tb{b)}\, 9 \frac{12}{40} \quad
\end{equation}

\vspace{1cm}

\begin{equation} 
0.875 =  \oh \tb{a)}\, \frac{7}{8} \quad \tb{b)}\, \frac{6}{8} \quad
\end{equation}

\vspace{1cm}

\end{multicols}

\setcounter{equation}{0}
Resuelve las siguientes operaciones con fracciones respetando su jerarqu\'ia y
selecciona la respuesta correcta. Simplifica cuando sea posible.

\begin{equation}
\frac{1}{3} + \frac{2}{3} + \frac{5}{3} = \oh 
\tb{a)}\, \frac{6}{9} \quad \tb{b)}\, 2 \quad
\end{equation}

\vspace{1.5cm}

\begin{equation}
\frac{2}{5} - \frac{4}{9} + \frac{14}{15} = 
\oh \tb{a)}\, 1\frac{7}{9} \quad \tb{b)}\, 1\frac{32}{45} \quad
\end{equation}

\vspace{1.5cm}

\begin{equation}
\frac{1}{2} + \frac{3}{5} \times \frac{2}{4} =
\oh \tb{a)}\, \frac{7}{8} \quad \tb{b)}\, \frac{12}{20} \quad 
\end{equation}

\vspace{1.5cm}

\begin{equation}
2\frac{1}{5} \div \frac{3}{2} \div \frac{1}{6} =
\oh \tb{a)}\, 8\frac{4}{5} \quad \tb{b)}\, \frac{133}{15} \quad 
\end{equation}

\vspace{1.5cm}

\begin{equation}
\frac{9}{5} \div \left( 4\frac{2}{6} - 1\frac{3}{12} \right) =
\oh \tb{a)}\, \frac{56}{54} \quad \tb{b)}\, \frac{54}{55} \quad 
\end{equation}

\vspace{1.5cm}

\begin{equation}
\left( \frac{5}{2} + \frac{3}{4} - \frac{1}{8} \right) \div 1\frac{2}{5} = 
\oh \tb{a)}\, \frac{56}{125} \quad \tb{b)}\, 2\frac{13}{56} \quad 
\end{equation}

\vspace{1.5cm}

\newpage
\setcounter{equation}{0}
Resuelve las siguientes operaciones con decimales respetando su jerarqu\'ia y
selecciona la respuesta correcta.


\begin{multicols}{2}

\begin{align}
&4.16 + 0.486 + 3 = \\
&\tb{a)}\, 7.646 \quad \tb{b)}\, 4.649 \quad \nonumber
\end{align}

\vspace{2cm}

\begin{align}
&4.7 - 0.826 = \\
&\tb{a)}\, 4.874 \quad \tb{b)}\, 3.874 \quad \nonumber
\end{align}

\vspace{2cm}

\begin{align}
&3.1416 \times 4.1 = \\
&\tb{a)}\, 12.88056 \quad \tb{b)}\, 128.7056 \quad \nonumber 
\end{align}

\vspace{2cm}

\begin{align}
&7.5 \div 0.25 = \\
&\tb{a)}\, 3.0 \quad \tb{b)}\, 30 \quad \nonumber 
\end{align}

\vspace{2cm}

\begin{align}
&3.2 + 4.8 \div 0.2 = \\
&\tb{a)}\, 27.2 \quad \tb{b)}\, 40 \quad \nonumber 
\end{align}

\vspace{2cm}

\begin{align}
&(3.2 + 4.8) \div 0.2 = \\
&\tb{a)}\, 27.2 \quad \tb{b)}\, 40 \quad \nonumber 
\end{align}

\vspace{2cm}

\end{multicols}

% \newpage
\vspace{2cm}

\section{Proporcionalidad}

Resuelve los problemas que se presentan a continuaci\'on y selecciona la
respuesta correcta.

\vspace{5mm}

Como promoci\'on en un supermercado se tiene que el costo de dos kilogramos de
az\'ucar es \$18.45. ?`Cu\'anto se deber\'a pagar por tres kilogramos y medio de
az\'ucar?

\hfill \tb{a)} \$322.8 \qquad \tb{b)} \$32.28

\vspace{5cm}

Dise\~nado por Volks Wagen,  el \textit{Bugatti Veyron} es el segundo auto m\'as
r\'apido del mundo alcanzando una velocidad m\'axima de 267.81 millas por hora.
Sabiendo que una milla por hora equivale a 1.609 kil\'ometros por hora,
determina dicha velocida en kil\'ometros por hora.

\hfill \tb{a)} 431\,Km/h \qquad \tb{b)} 429\,Km/h 

\vspace{4cm}

Para cierta venta especial en una tienda departamental se tienen la promoc\'on
de 18 mensualidades y 10\% acumulable en monedero electr\'onico. Aprovechando
esa oportunidad el profesor de matem\'aticas decidi\'o comprar una Macbook Pro
con precio de \$26,999.00. Calcula el costo de cada mensualidad y la cantidad
que se acumular\'a en el monedero.

\hfill \tb{a)} \$1,499.94 y \$2,699.9 \qquad \tb{b)} \$1,500.94 y \$269.99 

\vspace{4cm}

\section{Geometr\'ia}

Traza un tri\'angulo equil\'atero de $\ell=3$\,cm de longitud lateral. Mide su
altura; deber\'ia ser de 2.6\,cm. Usndo estos datos calcula su per\'imetro y su
\'area. Selecciona la respuesta correcta.

\hfill \tb{a)} 29\,cm, 78\,cm$^{2}$ \qquad \tb{b)} 9\,cm, 7.8\,cm$^{2}$

\newpage
Traza el esquema de un rect\'angulo cuya base mide $b=\frac{2}{3}$\,m y altura
mide $h=3.5$\,m. Calcula su per\'imetro y su \'area usando fracciones.
Selecciona la respuesta correcta.

\hfill \tb{a)} $8 \frac{1}{3}$\,m, $2 \frac{1}{3}$\,m$^{2}$ \qquad \tb{b)} 
$4\frac{1}{6}$\,m, $2 \frac{1}{6}$\,m$^{2}$

\vspace{5cm}
En la circunferencia que se encuentra a continuaci\'on marca sobre las lineas ya
dibujadas de color verde el radio y de color rojo el di\'ametro. Asumiendo que
el radio mide $r=1.5$\,cm. Calcula su per\'imetro y su\'area. Selecciona la
respuesta correcta.

\hfill \tb{a)} 4.7124\,cm, 22.20671376\,cm$^{2}$ \qquad 
\tb{b)} 9.4248\,cm, 7.0686\,cm$^{2}$


\begin{tikzpicture}
    \draw [thick, color=black] (0,0) circle (2cm);
    \draw [thick,color=black,] (0,0)--(2,0);
    \draw [thick,color=black,] (0,-2)--(0,2);
\end{tikzpicture}

\vspace{1.5cm}
Traza un cuadrado con longitud lateral $\ell=3.5$\,cm. Calcula su per\'imetro y
su \'area y selecciona la respuesta correcta.

\hfill \tb{a)} 150.0625\,cm, 12.25\,cm \qquad \tb{b)} 14\,cm, 12.25\,cm$^{2}$

\newpage
En el hex\'agono que se muestra a continuaci\'on marca con color verde la
l\'inea que corresponda al apotema. Asumiendo que su longitud lateral es
$\ell=2$\,Km y la de su apotema es $a=3.48$\,Km, calcula su per\'imetro y su
\'area. Selecciona la respuesta correcta.

\hfill \tb{a)} 12\,cm, 20.88\,cm$^{2}$ \qquad \tb{b)} 10\,cm, 17.4\,cm$^{2}$

\begin{tikzpicture}
\node [draw, thick, minimum size=4cm, regular polygon, regular polygon sides=6]
at (0,0) { };
\draw [thick,color=black,] (0,0)--(2,0);
\draw [thick,color=black,] (0,0)--(0,-1.74);
\end{tikzpicture}

\vspace{3mm}
Traza una l\'inea horizontal de 4\,cm y una vertical de 3\,cm de longitud. Traza
la mediatriz de cada una de ellas.

\vspace{4cm}
Traza un tri\'angulo de $45^{\circ}$ y uno de $25^{\circ}$. Traza la bisectriz
de cada uno de ellos.

\vspace{4cm}
A continuaci\'on se muestran dos figuras. Traza las bisectrices del tri\'angulo
y las mediatrices del cuadrado.

\vspace{3mm}
\begin{center}
\begin{tikzpicture}
\node [draw, thick, minimum size=5cm, regular polygon, regular polygon sides=3]
at (0,0) { };
\node [draw, thick, minimum size=5cm, regular polygon, regular polygon sides=4]
at (6,0.5) { };
\end{tikzpicture}
\end{center}
\vspace{3mm}

\newpage
Explica con tus palabras los conceptos de per\'imetro y \'area.

\vspace{3cm}
\section{Probabilidad y estad\'istica}

A continuaci\'on se presentan una serie de n\'umeros. Calcula su media
($\overline{x}$), moda ($m_{o}$) y mediana ($m_{e}$) y selecciona la respuesta
correcta.


17, 5, 5, 11, 10, 5, 17, 13, 4, 6, 3, 11, 6, 4, 11, 4, 8, 11, 15, 17, 5, 17, 16,
12

\hfill \tb{a)} $\overline{x}$=9.708  $m_{o}$=5, 11, 17  $m_{e}$= 10.5 \quad

\hfill \tb{b)} $\overline{x}$=9.78   $m_{o}$=5, 11, 17  $m_{e}$= 10, 11

\vspace{4cm}
Utilizando las edades de las personas que viven en tu casa, calcula la media y
la mediana. ?`Hay moda en las edades?

\vspace{3cm}
\section{Introducci\'on al \'Algebra}

Completa la tabla que se muestra a continuaci\'on separando las expresiones
algebraicas en sus respectivos componentes.

\begin{center}
{\large
\begin{tabular}{|c|c|c|c|}
\hline
Expresi\'on & Coeficiente & Literal(es) & Exponente(s)  \\ \hline 
$3x^2$ &&&\\ \hline
$b^2c^4x^6$ &&&\\ \hline
$17ax^2y^3$ &&&\\ \hline
$d^2we^3x$ &&&\\ \hline
\end{tabular}
}
\end{center}

\newpage
\setcounter{equation}{0}
Haz la reducci\'on de t\'erminos semejantes en las expresiones que se muestran a
continuaci\'on.

    
\begin{equation}
3m + 20p + 5m - 6p = 
\end{equation}
\begin{equation}
6p + 8m - 4p - 2m = 
\end{equation}
\begin{equation}
5p + m -p +2m + 6p =
\end{equation}
\begin{equation}
8p + m -5p + 12m + 2p + 2m =
\end{equation}
\begin{equation}
3x + 4y - x = 
\end{equation}
\begin{equation}
20x + 17y - 8x - y -3x =
\end{equation}

\vspace{7mm}
\setcounter{equation}{0}
Resuelve las siguientes ecuaciones.

\vspace{-5mm}
\begin{multicols}{2}

\begin{equation}    5x = 10     \end{equation}

\vspace{7mm}
\begin{equation}    15y = 15    \end{equation}

\vspace{7mm}
\begin{equation}    4z = 12     \end{equation}

\vspace{7mm}
\begin{equation}    7w = 14     \end{equation}

\vspace{7mm}
\begin{equation}    2a = 1      \end{equation}

\vspace{7mm}
\begin{equation}    15x = 90    \end{equation}

\vspace{7mm}
\begin{equation}    25y = 200   \end{equation}

\vspace{7mm}
\begin{equation}    5x = 10     \end{equation}

\vspace{7mm}
\end{multicols}

\newpage

Plantea las ecuaciones para los siguientes enunciados y resu\'elvelas. 

\vspace{5mm}

El doble de un n\'umero mas 5 es 15. ?`Cu\'al es ese n\'umero?

\vspace{2cm}
Cinco veces un n\'umero menos 10 es 40. ?`Cu\'al es ese n\'umero?

\vspace{2cm}
50 menos el triple de un n\'umero es 5. ?`Cu\'al es ese n\'umero?

\vspace{2cm}
Dos terceras partes de un n\'umero m\'as $\frac{3}{5}$ da como resultado
$\frac{12}{5}$. ?Cu\'al es ese n\'umero?

\vspace{2cm}
Se tiene una bolsa con canicas de color rojo y azul. Si se sabe que hay 17 rojas
y que en total hay 50 canicas, ?`Cu\'antas canicas azules hay?

\vspace{2cm}
Cada fin de semana Patricia recibe cierta cantidad de dinero para gastar. En
cierta ocasi\'on le dieron el doble y adem\'as encontr\'o \$20.00 en la calle.
Si antes de gastar el dinero logr\'o juntar \$180.00, ?`Cu\'anto es la cantidad
que le dan para gastar cada semana?


\end{document}

\left(  \right)




