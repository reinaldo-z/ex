\newcommand{\seccion}{SECUNDARIA INCORPORADA A LA SEG }
\newcommand{\descripcion}{Examen Diagn\'ostico de Matem\'aticas}
\newcommand{\grado}{Primero de secundaria}
\newcommand{\ciclo}{Ciclo escolar: 2015--2016}
\newcommand{\papel}{legalpaper} %letter, legalpaper ...

\documentclass[11pt]{article}
\usepackage[\papel]{geometry}

\title{\flushleft \seccion \\ \descripcion \\  \grado \\ \ciclo}

\newcommand\BackgroundLogo{
\put(162,365){
\parbox[b][\paperheight]{\paperwidth}{%
\vfill
\centering
\includegraphics[width=5cm,height=2.5cm,keepaspectratio]{/Users/reinaldo/Documents/clases/jassa/logo}%
\vfill
}}}



% \title{\seccion \\ \descripcion \\  \grado \\ \ciclo}

% \newcommand\BackgroundLogo{
% \put(162,435){
% \parbox[b][\paperheight]{\paperwidth}{%
% \vfill
% \centering
% \includegraphics[width=5cm,height=2.5cm,keepaspectratio]{/Users/reinaldo/Documents/clases/jassa/logo}%
% \vfill
% }}}


\newcommand{\fecha}{Septiembre de 2015}
\author{Ing. Arturo Canedo \\ M. en C. Reinaldo Zapata}


\hyphenation{con-ti-nua-ción}

\usepackage{enumitem}
\usepackage[T1]{fontenc} %fuentes
\usepackage{lmodern} %fuente mejorada
\usepackage[spanish]{babel}
\decimalpoint
\usepackage{fullpage}
\usepackage{multicol}
\usepackage{graphicx}
\usepackage{eso-pic}
\usepackage{multirow}
\usepackage{subfigure}


%Modificación del formato de las ecuaciones y el numerado de las mismas
\usepackage[leqno,fleqn]{amsmath} %[números a la izquierda, alineación de la ecuación a la izquierda]{mejora fórmulas}
\makeatletter
  \def\tagform@#1{\maketag@@@{#1\@@italiccorr}} %quita sección, paréntesis y da numeración continua
\makeatother
\renewcommand{\theequation}{\fbox{\textbf{\arabic{equation}}}} %números en negritas encerrados en rectángulo

% \renewcommand{\labelenumi}{\fbox{\arabic{enumi}}}



\begin{document}
\AddToShipoutPicture*{\BackgroundLogo}
\ClearShipoutPicture
\date{\fecha}
\maketitle
% \thispagestyle{empty}


Nombre del alumno:\,\line(1,0){244}\,.\hspace*{.2cm} No. de lista:\,\line(1,0){35}\,.

\grado, grupo:\,\line(1,0){30}\,.

\vspace{5mm}

El prop\'osito del examen diagn\'ostico es hacer una revisi\'on de los
conocimientos con los que llegan lso alumnos al grado de primero de secundaria.
Responde correctamente cada uno de los reactivos que se presentan a
continuaci\'on. Los resultados deber\'an ser escritos con tinta y sin
enmendaduras.

\section{N\'umeros}
Escribe las siguientes cantidades num\'ericas con letra sin faltas de
ortograf\'ia.
\vspace{0.7cm}

23 \leaders\vrule height 2pt depth -1.5pt \hfill \null\,.

\vspace{0.7cm}

648 \leaders\vrule height 2pt depth -1.5pt \hfill \null\,.

\vspace{0.7cm}

2769 \leaders\vrule height 2pt depth -1.5pt \hfill \null\,.

\vspace{1cm}

A continuaci\'on se dictar\'an unas cantidades num\'ericas. Escribe con n\'umero
cada una de ellas en los espacios que se muestran a continuaci\'on.

\begin{multicols}{2}
    \begin{equation} \text{\line(1,0){100}\,.} \end{equation}

    \begin{equation} \text{\line(1,0){100}\,.} \end{equation}

    \begin{equation} \text{\line(1,0){100}\,.} \end{equation}

    \begin{equation} \text{\line(1,0){100}\,.} \end{equation}
\end{multicols}



\setcounter{equation}{0} 

\section{Tablas de multiplicar {\normalsize(primera parte)}}
En los espacios que se muestran a continuaci\'on escribe el resultado de la
tabla de multiplicar que se dicta. El resultado debe escribirse al momento y no
hay oportunidad de escribirlo posteriormente, de lo contrario se anular\'a esta
secci\'on.

\begin{center}
\begin{multicols}{5}

    \begin{equation} \text{\line(1,0){45}\,.} \end{equation}

    \begin{equation} \text{\line(1,0){45}\,.} \end{equation}

    \begin{equation} \text{\line(1,0){45}\,.} \end{equation}

    \begin{equation} \text{\line(1,0){45}\,.} \end{equation}

    \begin{equation} \text{\line(1,0){45}\,.} \end{equation}

    \begin{equation} \text{\line(1,0){45}\,.} \end{equation}

    \begin{equation} \text{\line(1,0){45}\,.} \end{equation}

    \begin{equation} \text{\line(1,0){45}\,.} \end{equation}

    \begin{equation} \text{\line(1,0){45}\,.} \end{equation}

    \begin{equation} \text{\line(1,0){45}\,.} \end{equation}

    \begin{equation} \text{\line(1,0){45}\,.} \end{equation}

    \begin{equation} \text{\line(1,0){45}\,.} \end{equation}

    \begin{equation} \text{\line(1,0){45}\,.} \end{equation}

    \begin{equation} \text{\line(1,0){45}\,.} \end{equation}

    \begin{equation} \text{\line(1,0){45}\,.} \end{equation}

\end{multicols}
\end{center}

\setcounter{equation}{0} 

\section{Tablas de multiplicar {\normalsize(segunda parte)}}
De la misma manera como se hizo en la secci\'on anterior, escribe los
resultados de las tablas de multiplicar que se te indiquen.

\begin{center}
\begin{multicols}{5}

    \begin{equation} \text{\line(1,0){45}\,.} \end{equation}

    \begin{equation} \text{\line(1,0){45}\,.} \end{equation}

    \begin{equation} \text{\line(1,0){45}\,.} \end{equation}

    \begin{equation} \text{\line(1,0){45}\,.} \end{equation}

    \begin{equation} \text{\line(1,0){45}\,.} \end{equation}

    \begin{equation} \text{\line(1,0){45}\,.} \end{equation}

    \begin{equation} \text{\line(1,0){45}\,.} \end{equation}

    \begin{equation} \text{\line(1,0){45}\,.} \end{equation}

    \begin{equation} \text{\line(1,0){45}\,.} \end{equation}

    \begin{equation} \text{\line(1,0){45}\,.} \end{equation}

    \begin{equation} \text{\line(1,0){45}\,.} \end{equation}

    \begin{equation} \text{\line(1,0){45}\,.} \end{equation}

    \begin{equation} \text{\line(1,0){45}\,.} \end{equation}

    \begin{equation} \text{\line(1,0){45}\,.} \end{equation}

    \begin{equation} \text{\line(1,0){45}\,.} \end{equation}

\end{multicols}
\end{center}


\vspace{1cm}  

\section{Simplificaci\'on de fracciones}
Simplifica las siguientes fracciones hasta la m\'inima expresi\'on posible. Si
encuentras alguna que ya est\'e en su m\'inima expresi\'on, entonces ind\'icalo
y explica tu razonamiento.

\begin{multicols}{2}

\begin{equation}    \frac{4}{12}=   \nonumber\end{equation}

\begin{equation}    \frac{18}{36}=  \nonumber\end{equation}

\begin{equation}    \frac{3}{4}=    \nonumber\end{equation}

\begin{equation}    \frac{23}{31}=  \nonumber\end{equation}

\begin{equation}    \frac{25}{100}= \nonumber\end{equation}

\begin{equation}    \frac{12}{36}=  \nonumber\end{equation}

\end{multicols}

\vspace{1cm}

\section{Fracciones impropias y mixtas}
Convierte las fracciones mixtas a impropias y las fracciones impropias a mixtas.

\begin{multicols}{3}

\begin{equation}    \frac{15}{4}=   \nonumber\end{equation}

\begin{equation}    \frac{20}{3}=   \nonumber\end{equation}

\begin{equation}    \frac{25}{5}=   \nonumber\end{equation}

\begin{equation}    2\frac{2}{3}=   \nonumber\end{equation}

\begin{equation}    6\frac{5}{7}=   \nonumber\end{equation}

\begin{equation}    10\frac{4}{6}=  \nonumber\end{equation}

\end{multicols}

\vspace{0.5cm}

\section{Operaciones b\'asicas con fracciones}
Resuelve las siguientes operaciones con fracciones. Simplifica hasta la m\'inima
expresi\'on obteniendo enteros si es posible.

\begin{multicols}{2}

\begin{equation*}   \frac{2}{3}+\frac{4}{6}=        \end{equation*}

\begin{equation*}   5\frac{4}{7}+\frac{6}{14}=      \end{equation*}

\begin{equation*}   \frac{7}{2}-\frac{6}{5}=        \end{equation*}

\begin{equation*}   8-\frac{6}{5}=                  \end{equation*}

\begin{equation*}   \frac{1}{7}\times\frac{3}{4}=   \end{equation*}

\begin{equation*}   \frac{4}{9}\div\frac{2}{11}=    \end{equation*}

\end{multicols}

\vspace{0.7cm}

\section{Operaciones con enteros y decimales}
Resuelve las siguientes operaciones que incluyen n\'umeros enteros y decimales.
Haz el acomodo de las cantidades para poder efectuar cada operaci\'on.

\begin{multicols}{2}

\begin{equation*}    2.16 + 4. 897=          \end{equation*}

\begin{equation*}    42 + 0. 123=            \end{equation*}

\begin{equation*}    4.79 - 3.007=           \end{equation*}

\begin{equation*}    23 - 0.23=              \end{equation*}

\begin{equation*}    2.5 \times 3.1416=      \end{equation*}

\begin{equation*}    5.23 \div 1.7=          \end{equation*}

\end{multicols}

\vspace{1cm}

\section{Problemas}
Resuelve los problemas que se muestran a continuaci\'on.

\vspace{0.5cm}

En una reuni\'on familiar se repartieron las rebanadas de una pizza. Si la pizza
ten\'ia 8 rebanadas y s\'olo se comieron 6, ?`qu\'e fracci\'on de la pizza
sobr\'o?

\vspace{4cm}

El profesor de matem\'aticas decide hacer un peque\~no negocio con los alumnos
de su clase. Para ello compra una caja con 10 l\'apices por un costo de \$15.00.
?`Cu\'anto cuesta cada uno de ellos?

\vspace{4cm}

Si cada uno de los l\'apices lo vende a sus alumnos a un precio de \$2.50,
?`cu\'anto gan\'o despu\'es de haber vendido la caja completa?

\vspace{4cm}

Para el d\'ia en que este examen se est\'a aplicando (\fecha) se
prev\'e que la temperatura m\'inima sea de 16$^{\circ}$\,C y la temperatura
m\'axima de 27$^{\circ}$\,C. ?`Cu\'al es la diferencia entre ambas temperaturas?

\vspace{4cm}

Para un torneo de b\'asquetbol se tiene un total de 45 participantes. Si el
reglamento dice que los equipos pueden tener un m\'aximo de 8 integrantes,
?`cu\'antos equipos se pueden formar?

\vfill

{\footnotesize \hfill Este examen fue escrito en \LaTeX, un software libre, gratuito}

{\footnotesize \hfill y profesional para la edici\'on de documentos. Si est\'as }

{\footnotesize \hfill interesado en saber m\'as, pregunta a tu profesor.}



\end{document}
