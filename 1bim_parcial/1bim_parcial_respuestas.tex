\newcommand{\seccion}{SECUNDARIA INCORPORADA A LA SEG }
\newcommand{\descripcion}{Examen Parcial, Primer Bimestre}
\newcommand{\grado}{Primero de secundaria}
\newcommand{\ciclo}{Ciclo escolar: 2015--2016}
\newcommand{\papel}{legalpaper} %letter, legalpaper ...
\newcommand{\fecha}{5 de octubre de 2015}
% \author{Ing. Arturo Canedo \\ M. en C. Reinaldo Zapata}
\author{M. en C. Reinaldo Zapata}

\documentclass[11pt]{article}
\usepackage[\papel]{geometry}

\title{\flushleft \seccion \\ \descripcion \\  \grado \\ \ciclo}

\newcommand\BackgroundLogo{
\put(162,365){
\parbox[b][\paperheight]{\paperwidth}{%
\vfill
\centering
\includegraphics[width=5cm,height=2.5cm,keepaspectratio]{/Users/reinaldo/Documents/clases/jassa/logo}%
\vfill
}}}



% \title{\seccion \\ \descripcion \\  \grado \\ \ciclo}

% \newcommand\BackgroundLogo{
% \put(162,435){
% \parbox[b][\paperheight]{\paperwidth}{%
% \vfill
% \centering
% \includegraphics[width=5cm,height=2.5cm,keepaspectratio]{/Users/reinaldo/Documents/clases/jassa/logo}%
% \vfill
% }}}

\hyphenation{con-ti-nua-ción}

\usepackage{enumitem}
\usepackage[T1]{fontenc} %fuentes
\usepackage{lmodern} %fuente mejorada
\usepackage[spanish]{babel}
\decimalpoint
\usepackage{fullpage}
\usepackage{multicol}
\usepackage{graphicx}
\usepackage{eso-pic}
\usepackage{multirow}
\usepackage{subfigure}


%Modificación del formato de las ecuaciones y el numerado de las mismas
\usepackage[leqno,fleqn]{amsmath}
\makeatletter
  \def\tagform@#1{\maketag@@@{#1\@@italiccorr}}
\makeatother
\renewcommand{\theequation}{\fbox{\textbf{\arabic{equation}}}}


\begin{document}
\AddToShipoutPicture*{\BackgroundLogo}
\ClearShipoutPicture
\date{\fecha}
\maketitle
% \thispagestyle{empty}


Nombre del alumno:\,\line(1,0){244}\,.\hspace*{.2cm} No. de lista:\,\line(1,0){35}\,.

\grado, grupo:\,\line(1,0){30}\,.

\vspace{5mm}

El prop\'osito de todo examen es poner a prueba los conocimientos de cada
alumno para calificar as\'i su desempe\~no y aprendizaje. Contesta
correctamente, en cada secci\'on, tantos reactivos como te sea posible. Todas y
cada una de las respuestas deber\'as escribirlas en esta hoja donde se
imprimi\'o el examen.


\section{M\'inimo com\'un m\'ultiplo}
Encuentra el m\'inimo com\'un m\'ultiplo de cada uno de los grupos de n\'umeros
que se muestran a continuaci\'on (2 aciertos).

\begin{multicols}{2}

\begin{equation*} 1, 2, 3, 5, 6, 15, 30    \end{equation*}

\vspace{5mm}

\quad MCM=30

\vspace{1.5cm}

\begin{equation*} 14, 5, 7, 11, 10,    \end{equation*}

\vspace{5mm}

\quad MCM=770

\vspace{1.5cm}


\begin{equation*} 50, 200, 100, 150    \end{equation*}

\vspace{5mm}

\quad MCM=600

\vspace{1.5cm}


\begin{equation*} 12, 6, 18, 2, 9    \end{equation*}

\vspace{5mm}

\quad MCM=36

\vspace{1.5cm}


\end{multicols}

\vspace{2cm}

\section{Conversi\'on de fracciones mixtas a impropias e impropias a mixtas}
Seg\'un sea el caso, convierte la fracci\'on mixta a impropia o la impropia a
mixta, para los siguientes incisos (6 aciertos).

\begin{multicols}{3}
    \begin{equation*} 5\frac{1}{2}= \frac{11}{2} \end{equation*}

    \begin{equation*} 3\frac{4}{7}= \frac{25}{7} \end{equation*}

    \begin{equation*} 3\frac{4}{10}= \frac{34}{10} \end{equation*}

    \begin{equation*} \frac{9}{3}= 3  \end{equation*}

    \begin{equation*} \frac{8}{5}= 1\frac{3}{5}  \end{equation*}

    \begin{equation*} \frac{15}{4}= 3\frac{3}{4} \end{equation*}
\end{multicols}

\vspace{1cm}



\section{Fracciones}

?`Cu\'al es el proceso para hacer una suma o resta de fracciones? Expl\'icalo
con tus propias palabras.

\vspace{5mm}

Para sumar o restar fracciones es necesario obtener el m\'inimo con\'um
m\'ultiplo de los denominadores. Despu\'es se divide este resultado entre cada
uno de los denominadores y se multiplica por el numerador correspondiente.

\vspace{1.5cm}

Explica con tus palabras que es una fracci\'on propia, impropia y mixta.\\

Propia: fracci\'on en la que el numerador es m\'as peque\~no que el denominador.

\vspace{1.5cm}

Impropia: fracci\'on en la que el numerador es m\'as grande que el denominador.

\vspace{1.5cm}

Mixta: fracci\'on compuesta por un n\'umero entero y una fracci\'on propia.

\vspace{1.5cm}


Simplifica las siguientes fracciones hasta su m\'inima expresi\'on obteniendo
enteros si es posible. En caso de que no se pueda simplificar explica tu
razonamiento.

\begin{multicols}{2}

\begin{equation*}    \frac{4}{12}= \frac{1}{3}  \end{equation*}

\begin{equation*}    \frac{18}{36}= \frac{1}{2} \end{equation*}

\begin{equation*}    \frac{3}{4}= \frac{3}{4}   \end{equation*}

\begin{equation*}    \frac{23}{31}= \frac{23}{21} \end{equation*}

\begin{equation*}    \frac{25}{100}= \frac{1}{4} \end{equation*}

\begin{equation*}    \frac{12}{36}= \frac{1}{3} \end{equation*}

\end{multicols}

\vspace{1cm}

Resuelve cada una de las operaciones con fracciones. Simplifica el resultado
hasta su m\'inima expresi\'n.
    

\begin{equation*}
    \frac{2}{3} + 4 + 2 \frac{1}{7}= \frac{143}{21} = 6\frac{17}{21} 
\end{equation*}

\vspace{2.5cm}

\begin{equation*}
    12\frac{1}{16} - \frac{7}{8} - \frac{5}{32}  = \frac{353}{32} = 11\frac{1}{32}
\end{equation*}

\vspace{2.5cm}

\begin{equation*}
    \frac{3}{10} \times 2\frac{1}{4} = \frac{27}{40}
\end{equation*}

\vspace{2.5cm}

\begin{equation*}
    7 \div \frac{2}{7} = \frac{49}{2} = 24\frac{1}{2}
\end{equation*}

\vspace{2.5cm}

\section{Teor\'ia}

Escribe las reglas de divisibilidad de dos n\'umeros (4 aciertos).

\vspace{2.5cm}

Acorde a lo visto en clase, acomoda los siguientes incisos de la izquierda en
las l\'ineas que est\'an a la derecha acorde al orden en que se tienen que
realizar las operaciones.

\vspace{1cm}

\begin{minipage}[t]{0.5\textwidth}
Multplicaci\'on y divisi\'on.

Potencias y ra\'ices: $r^{2}$, $4^{2}$, $\sqrt{2}$, $\sqrt{25}$.

Suma y resta.

Signos de agrupaci\'on: (), [], \{\}.

\end{minipage}
\begin{minipage}[t]{0.5\textwidth}

1.- Signos de agrupaci\'on: (), [], \{\}.

\vspace{3mm}

2.- Potencias y ra\'ices: $r^{2}$, $4^{2}$, $\sqrt{2}$, $\sqrt{25}$.

\vspace{3mm}

3.- Multplicaci\'on y divisi\'on.

\vspace{3mm}

4.- Suma y resta.

\vspace{3mm}


\end{minipage}

\section{Problemas}

Resuelve los problemas qeu se plantean a continuaci\'on. \\


Para una parrillada se compraron $3\frac{2}{3} $\,Kg de arrachera  marinada,
$2\frac{1}{2} $\,Kg de chorizo tradicional, $\frac{3}{4} $\,Kg de chorizo
argentino y 4\,Kg de costilla para asar. ?`Qu\'e cantidad total de carne (en Kg)
se compr\'o para dicha parillada?

\vspace{5mm}

\begin{equation*}
3\frac{2}{3} + 2\frac{1}{2} + \frac{3}{4} + 4 = \frac{131}{12} = 10\frac{11}{12}\text{\,Kg}
\end{equation*}

\vspace{5mm}


Si el costo de la arrachera marinada es de \$160.00 por Kilogramo, ?`Cu\'anto se
pag\'o s\'olo por la arrachera?

\vspace{5mm}

\begin{equation*}
3\frac{2}{3} \times \frac{160}{1} = \frac{1760}{3} = 586\frac{2}{3}  \approx \$586.60
\end{equation*}


\end{document}
