\newcommand{\seccion}{SECUNDARIA INCORPORADA A LA SEG }
\newcommand{\descripcion}{Examen Bimestral, Segundo Bimestre }
\newcommand{\grado}{Primero de secundaria}
\newcommand{\ciclo}{Ciclo escolar: 2015--2016}
\newcommand{\papel}{legalpaper} %letterpaper, legalpaper ...
\newcommand{\fecha}{15 de diciembre de 2015}

\author{M. en C. Reinaldo Zapata}

\documentclass[11pt]{article}
\usepackage[\papel]{geometry}

\title{\flushleft \seccion \\ \descripcion \\  \grado \\ \ciclo}

\newcommand\BackgroundLogo{
\put(162,365){
\parbox[b][\paperheight]{\paperwidth}{%
\vfill
\centering
\includegraphics[width=5cm,height=2.5cm,keepaspectratio]{/Users/reinaldo/Documents/clases/jassa/logo}%
\vfill
}}}

% \hyphenation{con-ti-nua-ci\'on}

\usepackage{enumitem}
\usepackage[T1]{fontenc} %fuentes
\usepackage{lmodern} %fuente mejorada
\usepackage[spanish]{babel}
\decimalpoint
\usepackage{fullpage}
\usepackage{multicol}
\usepackage{graphicx}
\usepackage{eso-pic}
\usepackage{multirow}
\usepackage{subfigure}
\usepackage{tikz}
\usepackage{hyperref} 
\usepackage{color}
\usepackage{multicol}
\usepackage{tikz}
\usetikzlibrary{shapes.geometric}



\usepackage[leqno,fleqn]{amsmath}
\makeatletter
  \def\tagform@#1{\maketag@@@{#1\@@italiccorr}}
\makeatother
\renewcommand{\theequation}{\fbox{\textbf{\arabic{equation}}}}


\begin{document}
\AddToShipoutPicture*{\BackgroundLogo}
\ClearShipoutPicture
\date{\fecha}
\maketitle


% \begin{minipage}[t]{0.8\linewidth}
Nombre del alumno:\,\line(1,0){244}\,.\hspace*{.2cm} \hfill Aciertos:\,\line(1,0){35}\,. \\
\indent Primero de secundaria, grupo:\,\line(1,0){35}\,. No. de lista:\,\line(1,0){35}\,. \hfill 40 \quad \ 
% \end{minipage}
% \begin{minipage}{0.8\linewidth}
% \end{minipage}

\vspace{5mm}

El prop\'osito de todo examen es poner a prueba los conocimientos de cada alumno
para calificar as\'i su desempe\~no y aprendizaje. Contesta correctamente, en
cada secci\'on, tantos reactivos como te sea posible. Todas y cada una de las
operaciones deber\'as escribirlas en esta hoja donde se imprimi\'o el examen y
los resultados finales deber\'an estar escritos con tinta.

\section{Fracciones y decimales}

Convierte cada fracci\'on a decimal y viceversa. Simplifica tus resultados si es
posible. \hfill \textbf{(2pts.)} 
\begin{multicols}{2}

\begin{equation*} 4\frac{5}{16} = 4.31 \end{equation*}

\begin{equation*} 3.1416 = 3 \frac{1416}{10000} = 3 \frac{177}{1250} \end{equation*}

\end{multicols}

\vspace{5mm}

Resuelve las siguientes operaciones con fracciones respetando la jerarqu\'ia de
las operaciones.

\hfill \textbf{(6pts.)}
\begin{equation*}
\left(  \frac{2}{3} + \frac{2}{15} \right) \div \frac{1}{5} = \frac{60}{15} = 4
\end{equation*}

\vspace{1.5cm}

\begin{equation*}
\frac{13}{5} \div \left( \frac{2}{9} + \frac{5}{18} \right) = \frac{234}{45} = 5 \frac{1}{5}
\end{equation*}

\vspace{1.5cm}

Resuelve las siguientes operaciones con decimales respetando la jerarqu\'ia de
las operaciones. 

\hfill \textbf{(4pts.)}
\begin{multicols}{2}
\begin{equation*}
(4.5 + 8)(3.4) = 42.5
\end{equation*}

\begin{equation*}
7 + 11 \times 6.5 = 78.5
\end{equation*}
\end{multicols}

\vspace{5cm}

Resuelve el problema. 

Hilari\'on S\'anchez, famoso ciclista de carreras Leon\'es, entrena a sus
pupilos en el Parque Metropolitano. El primer d\'ia dan 10 vueltas, el segundo
11$\frac{2}{5}$ vueltas y el tercero 12.6 vueltas. ?`Cu\'antas vueltas
recorrieron en total durante estos d\'ias? Expresa el resultado en decimales.
\hfill \textbf{(2pts.)}

\vspace{1.5cm}

\begin{equation*}
10 + 11.4 + 12.6 = 34. \qquad\qquad\qquad\qquad\qquad \text{ 34 vueltas.}
\end{equation*}

\vspace{1.5cm}

\section{Ubicaci\'on espacial}

Usando la recta que se muestra a continuaci\'on coloca en el extremo izquierdo
el n\'umero uno (1) y en el extremo derecho el n\'umero tres (3). Usando esa
misma recta ubica las siguientes cantidades: 

\hfill \textbf{(5pts.)}

\begin{multicols}{5}
\begin{enumerate}[label=\alph*)] \itemsep-.3em
\item $\displaystyle\frac{4}{5} $ \hspace{5mm} 
\item 1.9 
\item 2.6 
\item 1.1
\item $\displaystyle\frac{11}{6} $
\end{enumerate}
\end{multicols}

\vspace{5mm}

\begin{centering}

\begin{tikzpicture}
\draw [thick,color=black] (-1,0)--(10,0);
\node [below] at (0,0) {1};
\node [below] at (5,0) {2};
\node [below] at (10,0) {3};
\node [above] at (-1,0) {a};
\node [above] at (4.5,0) {b};
\node [above] at (8,0) {c};
\node [above] at (4,0) {e};
\node [above] at (0.5,0) {d};\end{tikzpicture}

\end{centering}

\section{Probabiliad y estad\'istica}

En una bolsa hay un total de quince (15) canicas de las cuales cinco (5) son de
color amarillo y diez (10) son de color verde. Responde a las siguientes
preguntas: \hfill  \textbf{(4pts.)}

\begin{enumerate}[label=\alph*)] \itemsep-.3em
\item ?`Qu\'e fracci\'on del total representan las canicas verdes? \qquad $\frac{10}{15} = \frac{2}{3}$ 
\item ?`Qu\'e fracci\'on del total representan las canicas amarillas? \qquad $\frac{5}{15} = \frac{1}{3}$ 
\item ?`Qu\'e color de canica es m\'as probable que salga si se saca una canica
al azar de la bolsa? Explica tu respuesta.

\qquad Las de color verde porque son m\'as.
\end{enumerate}

\vspace{10mm}

Recuerda el experimento de lanzar dos dados. Acorde a ello responde para cada
inciso si la probabilidad de que suceda el evento es alta (\emph{A}), baja
(\emph{B}) o nula (0). \hfill  \textbf{(3pts.)}

\begin{enumerate}[label=\alph*)] \itemsep-.3em
\item Que la suma de los n\'umeros en los dados sea 1 o 13. \qquad {0}
\item Que la suma de los n\'umeros de los dados sea 6, 7 u 8. \qquad {A}
\item Que la suma de los n\'umeros de los dados sea 2 o 12. \qquad {B}
\end{enumerate}

\section{Geometr\'ia}
    
Traza un cuadrado que tenga 4.5\,cm de longitud lateral. Calcula su per\'imetro y
su \'area.

\hfill  \textbf{(3pts.)}

\begin{minipage}{0.3\linewidth}
    
\begin{tikzpicture}[scale=2]
\node [draw, thick, minimum size=4.5cm, regular polygon, regular polygon sides=4] at (0,0) { };
\end{tikzpicture}

P=18\,cm

A=20.25\,cm$^{2}$
\end{minipage}
\begin{minipage}[t]{0.3\linewidth}
\begin{align*}
P =& 4 \ell \\
P =& (4) (4.5) \\
P =& 18\\\\
\end{align*}
\end{minipage}
\begin{minipage}[t]{0.3\linewidth}
\begin{align*}
A =& \ell^{2} \\
A =& (4.5)^{2} \\
A =& 20.25 
\end{align*}

\end{minipage}

\vspace{1cm}


Usando el siguiente hex\'agono remarca la l\'inea que corresponde al apotema.
Escribe las f\'ormulas para calcular su per\'imetro y su \'area. Utilizando como
medida del apotema $a=2.7$\,cm y como longitud lateral $\ell=3$\,cm, calcula su
per\'imetro y su \'area. \hfill  \textbf{(5pts.)}

\begin{minipage}{0.4\linewidth}
\begin{tikzpicture}[scale=2]
\node [right] at (0,1.3) { };
\node [draw, thick, minimum size=5cm, regular polygon, regular polygon sides=6] at (0,0) { };
\draw [thick,color=red,] (0,0)--(0,-1.07);
\end{tikzpicture}    

P=18\,cm

A=24.3\,cm$^2$
\end{minipage}
\begin{minipage}{0.3\linewidth}
\begin{align*}
P =& n \ell \\
P =& (6)(3) \\
P =& 18
\end{align*}
\end{minipage}
\begin{minipage}{0.3\linewidth}
\begin{align*}
A =& \frac{Pa}{2} \\
A =& \frac{(18)(2.7)}{2} \\
A =& 24.3
\end{align*}    
\end{minipage}

\vspace{5mm}
% \vspace{1cm}

Utilizando las l\'ineas que se muestran a continuaci\'on traza un tri\'angulo.
Recuerda usar el comp\'as. \hfill  \textbf{(2pts.)}

\vspace{1cm}
\begin{minipage}{0.5\linewidth}
% \begin{figure}[h]
\begin{tikzpicture}[scale=3]
\draw [thick,color=black] (0,0.5)--(1,0.5);
\draw [thick,color=black] (0,0.25)--(1.5,0.25);
\draw [thick,color=black] (0,0)--(2,0);
\end{tikzpicture}
% \end{figure}
\end{minipage}
\begin{minipage}{0.5\linewidth}
\begin{tikzpicture}[scale=3]
\draw [thick,color=black] (0,0)--(2,0);
\draw [thick,color=black] (0,0) -- ++(28.955:1.447);
\draw [thick,color=black] (2,0) -- ++(136.567:1.017);
\end{tikzpicture}
\end{minipage}

\vspace{4cm}

Traza las mediatrices de los siguientes segmentos y la bisectriz de los
siguientes \'angulos. \hfill  \textbf{(2pts.)}

\vspace{1cm}

\begin{center}
\begin{tikzpicture}[scale=0.9,]%rotate=180
\draw [thick,color=black] (0,2)--(4,0);
% \draw [thick,color=red] (2,1)--(1,-2);
\draw [thick,color=black] (4,2)--(7,0); \draw [thick,color=black] (4,2)--(5,-1);
\draw [thick,color=black] (8,0)--(12,2);
\draw [thick,color=black] (11,-1)--(14,2); \draw [thick,color=black] (14,2)--(15,-1.2) ;
\end{tikzpicture}
\end{center}

\section{Proporcionalidad} % (fold)
\label{sec:proporcionalidad}
Resuelve el siguiente problema.

Una motocicleta recorre una distancia de 10\,Km con 2 litros de gasolina.
?`Qu\'e distancia podr\'a recorrer si se llena el tanque con 15 litros? \hfill
\textbf{(2pts.)}

\begin{center}
\begin{tabular}{ccc}
10\,Km & \qquad\qquad & 2$\ell$ \\
$x$? && 15$\ell$
\end{tabular}

\vspace{5mm}

\begin{align*}
    x =& (15)(10)\div 2 \\
    x =& 75\text{\,Km}
\end{align*}


\end{center}

% section proporcionalidad (end)


\end{document}

\left(  \right)




