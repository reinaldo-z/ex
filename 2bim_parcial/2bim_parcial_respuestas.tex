\newcommand{\seccion}{SECUNDARIA INCORPORADA A LA SEG }
\newcommand{\descripcion}{Examen Parcial, Segundo Bimestre}
\newcommand{\grado}{Primero de secundaria}
\newcommand{\ciclo}{Ciclo escolar: 2015--2016}
\newcommand{\papel}{legalpaper} %letter, legalpaper ...
\newcommand{\fecha}{24 de noviembre de 2015}
% \author{Ing. Arturo Canedo \\ M. en C. Reinaldo Zapata}
\author{M. en C. Reinaldo Zapata}

\documentclass[11pt]{article}
\usepackage[\papel]{geometry}

\title{\flushleft \seccion \\ \descripcion \\  \grado \\ \ciclo}

\newcommand\BackgroundLogo{
\put(162,365){
\parbox[b][\paperheight]{\paperwidth}{%
\vfill
\centering
\includegraphics[width=5cm,height=2.5cm,keepaspectratio]{/Users/reinaldo/Documents/clases/jassa/logo}%
\vfill
}}}



% \title{\seccion \\ \descripcion \\  \grado \\ \ciclo}

% \newcommand\BackgroundLogo{
% \put(162,435){
% \parbox[b][\paperheight]{\paperwidth}{%
% \vfill
% \centering
% \includegraphics[width=5cm,height=2.5cm,keepaspectratio]{/Users/reinaldo/Documents/clases/jassa/logo}%
% \vfill
% }}}

\hyphenation{con-ti-nua-ción}

\usepackage{enumitem}
\usepackage[T1]{fontenc} %fuentes
\usepackage{lmodern} %fuente mejorada
\usepackage[spanish]{babel}
\decimalpoint
\usepackage{fullpage}
\usepackage{multicol}
\usepackage{graphicx}
\usepackage{eso-pic}
\usepackage{multirow}
\usepackage{subfigure}
\usepackage{tikz}


%Modificación del formato de las ecuaciones y el numerado de las mismas
\usepackage[leqno,fleqn]{amsmath}
\makeatletter
  \def\tagform@#1{\maketag@@@{#1\@@italiccorr}}
\makeatother
\renewcommand{\theequation}{\fbox{\textbf{\arabic{equation}}}}


\begin{document}
\AddToShipoutPicture*{\BackgroundLogo}
\ClearShipoutPicture
\date{\fecha}
\maketitle
% \thispagestyle{empty}


Nombre del alumno:\,\line(1,0){244}\,.\hspace*{.2cm} No. de lista:\,\line(1,0){35}\,.

\grado, grupo:\,\line(1,0){30}\,.

\vspace{5mm}

El prop\'osito de todo examen es poner a prueba los conocimientos de cada alumno
para calificar as\'i su desempe\~no y aprendizaje. Contesta correctamente, en
cada secci\'on, tantos reactivos como te sea posible. Todas y cada una de las
operaciones deber\'as escribirlas en esta hoja donde se imprimi\'o el examen y
los resultados finales deber\'an estar escritos con tinta.


\section{Operaciones y problemas con fracciones}

Resuelve las siguientes operaciones con fracciones 


\begin{equation*}
    \left( \frac{1}{2} - \frac{1}{3} \right) (6) = \frac{6}{6} =  1
\end{equation*}

\vspace{2cm}

\begin{equation*}
    \left( 5\frac{1}{4} - 4 \right) \div 1\frac{1}{2} = \frac{10}{12} =  \frac{5}{6}
\end{equation*}

\vspace{2cm}

?`Cu\'antas varillas de $\frac{1}{4}$ de metro de longitud se pueden sacar de una varilla de 2 metros?

\vspace{1cm}

\hfill 8 varillas \hfill \ 

\vspace{1cm}

Un hombre camina 4$\frac{1}{2}$\,Km el lunes, 8$\frac{2}{3}$\,Km el martes, 10\,Km el mi\'ercoles y $\frac{5}{8}$\,Km el jueves. ?`Cu\'anto ha recorrido en los cuatro d\'ias ? 

\vspace{1cm}

\hfill $\displaystyle\frac{575}{24}$\,Km = $23\displaystyle\frac{19}{24}$\,Km \hfill \ 

\newpage

\section{Decimales, per\'imetros y \'areas }

Resuelve las siguientes operaciones con decimales.

\begin{multicols}{2}

\begin{equation*}
    78 - 69.45 + 4.5= 13.05
\end{equation*}

\begin{equation*}
    8.15 \div 4.3 = 1.895
\end{equation*}

\end{multicols}

\vspace{3cm}

Traza el esquema de un c\'irculo cuyo radio es $r=50$\,m. Calcula su per\'imetro y su \'area.

\vspace{5mm}

\begin{minipage}{0.5\linewidth}
\begin{tikzpicture}
\draw[black,thick] (2,2) circle (2cm);
\draw[thick,-] (2,2) -- (4,2) node[anchor=west] {r=50m};
\end{tikzpicture}
\end{minipage}%
\begin{minipage}{0.5\linewidth}
\begin{align*}
P =& \pi D \\
P =& (3.1416)(100 \text{\,m}) \\
P =& 314.16 \text{\,m} \\ \\
A =& \pi r^{2} \\
A =& (3.1416)(50\text{\,m})(50\text{\,m}) \\
A =& 7,854\text{\,m}^{2} 
\end{align*}
\end{minipage}

\vspace{5mm}

Un trabajador es contratado para pintar una pared cuadrada cuyo lado mide 4.3\,m. Si el costo de pintura es de \$100 por metro cuadrado, ?`Cu\'al ser\'a el costo de pintar dicha pared?

\vspace{5mm}

\begin{minipage}[t]{0.5\linewidth}
Primero se calcula el \'area:
\begin{align*}
A =& \ell^{2} \\
A =& (4.3\text{\,m})(4.3\text{\,m})\\
A =& 18.49\text{\,m}^{2}
\end{align*}
\end{minipage}%
\begin{minipage}[t]{0.5\linewidth}
Despu\'es se calcula el costo:
\begin{align*}
\text{Costo} =& (\text{\'Area})(\text{Precio}) \\
\text{Costo} =& (18.49\text{\,m}^{2})(\$100/\text{m}^{2})\\
\text{Costo} =& \$1849.00
\end{align*}
\end{minipage}
\vspace{5mm}

Escribe las f\'ormulas del per\'imetro y el \'area de las siguientes figuras:

\vspace{1cm}

\begin{minipage}[t]{0,5\linewidth}
\begin{center}
    Tri\'angulo
    \begin{align*}
    P =& 3\ell \\
    A =& \frac{bh}{2}
    \end{align*}
\end{center}
\end{minipage}
\begin{minipage}[t]{0,5\linewidth}
\begin{center}
    Pent\'agono
    \begin{align*}
    P =& n\ell = 5\ell \\
    A =& \frac{Pa}{2}
    \end{align*}
\end{center}
\end{minipage}


\end{document}






