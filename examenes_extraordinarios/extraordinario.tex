\newcommand{\seccion}{SECUNDARIA INCORPORADA A LA SEG }
\newcommand{\descripcion}{Examen Extraordinario de  Matem\'aticas }
\newcommand{\grado}{Primero de secundaria}
\newcommand{\ciclo}{Ciclo escolar: 215--2016 }
\newcommand{\papel}{legalpaper} %letter, legalpaper ...

\documentclass[11pt]{article}
\usepackage[\papel]{geometry}
\title{\flushleft \seccion \\ \descripcion \\  \grado \\ \ciclo}

% \title{\flushleft SECUNDARIA INCORPORADA A LA SEG \\ Planeaci\'on del periodo de Recuperaci\'on \\ }

\newcommand{\fecha}{Lunes 14 de septiembre de 2015}
\author{M. en C. Reinaldo Arturo Zapata Pe\~na}


\hyphenation{con-ti-nua-ción}

\usepackage{enumitem}
\usepackage[T1]{fontenc} %fuentes
\usepackage{lmodern} %fuente mejorada
\usepackage[spanish]{babel}
\decimalpoint
\usepackage{fullpage}
\usepackage{multicol}
\usepackage{graphicx}
\usepackage{eso-pic}
\usepackage{multirow}
\usepackage{subfigure}

\newcommand\BackgroundLogo{
\put(162,365){
\parbox[b][\paperheight]{\paperwidth}{%
\vfill
\centering
\includegraphics[width=5cm,height=2.5cm,keepaspectratio]{/Users/reinaldo/Documents/clases/jassa/logo}%
\vfill
}}}

%Modificación del formato de las ecuaciones y el numerado de las mismas
\usepackage[leqno,fleqn]{amsmath} %[números a la izquierda, alineación de la ecuación a la izquierda]{mejora fórmulas}
\makeatletter
  \def\tagform@#1{\maketag@@@{#1\@@italiccorr}} %quita sección, paréntesis y da numeración continua
\makeatother
\renewcommand{\theequation}{\fbox{\textbf{\arabic{equation}}}} %números en negritas encerrados en rectángulo

% \renewcommand{\labelenumi}{\fbox{\arabic{enumi}}}



\begin{document}
\AddToShipoutPicture*{\BackgroundLogo}
\ClearShipoutPicture
\date{\fecha}
\maketitle
% \thispagestyle{empty}

Nombre del alumno:\,\line(1,0){244}\,.\hspace*{.2cm} No. de lista:\,\line(1,0){35}\,.

Grado, grupo:\,\line(1,0){50}\,.

\vspace{1cm}

Este examen abarca los temas vistos a lo largo del curso de Matem\'aticas de primero de secundaria. Contesta de forma correcta conforme se indica en cada secci\'on. Recuerda escribir todas y cada una de las operaciones y procedimientos en estas mismas hojas. Esribe con tinta la respuesta final.


\vspace{1cm}

\section{N\'umeros fraccionarios}

Completa las fracciones equivalentes que se muestran a continuaci\'on

\begin{multicols}{3}

\begin{equation}    8 = \frac{\text{\hspace{7mm}}}{5}    \end{equation}

\begin{equation}    4 = \frac{\text{\hspace{7mm}}}{4}    \end{equation}

\begin{equation}    6 = \frac{\text{\hspace{7mm}}}{8}    \end{equation}

\begin{equation}    \frac{6}{10}     = \frac{\text{\hspace{7mm}}}{5}        \end{equation}

\begin{equation}    \frac{9}{24}     = \frac{\text{\hspace{7mm}}}{8}        \end{equation}

\begin{equation}    \frac{10}{18}    = \frac{\text{\hspace{7mm}}}{9}        \end{equation}


\end{multicols}

\vspace{1cm}

Simplifica las siguientes fracciones hasta su m\'inima expresi\'on obteniendo enteros si es posible.
\setcounter{equation}{0}


\begin{multicols}{3}

\begin{equation}    \frac{154}{96}    =     \end{equation}

\begin{equation}    \frac{60}{90}    =     \end{equation}

\begin{equation}    \frac{24}{12}    =     \end{equation}

\end{multicols}

\vspace{1cm}

Resuelve las siguientes operaciones con fracciones. Simplifica el resultado hasta su m\'inima expresi\'on obteniendo enteros si es posible.
\setcounter{equation}{0}


\begin{equation}    9 - \frac{10}{16} +\frac{5}{12} =   \end{equation} \\

\begin{equation}    80 - 3\frac{4}{5} -4\frac{2}{10} =  \end{equation} \\

\begin{equation}    \left( 8 +  \frac{3}{4} \right) \div 4\frac{1}{5}   =   \end{equation}\\

\begin{multicols}{2}

\begin{equation}    10 \times \frac{70}{39} = \frac{700}{39} =      \end{equation}\\

\begin{equation}    \frac{6}{9} \div \frac{12}{7} = \end{equation}\\

\begin{equation}    \left( \frac{3}{2} \right)^3= \end{equation} \\

\begin{equation}    \sqrt{\frac{16}{81}}= \end{equation} \\

\end{multicols}

\section{Equivalencia entre fracciones y decimales}

Convierte los n\'umeros fraccionarios a decimales y los decimales a fraccionarios.
\setcounter{equation}{0}


\begin{multicols}{2}

\begin{equation}    5\frac{5}{10} = \end{equation}
\vspace{1cm}

\begin{equation}    \frac{4}{12} = \end{equation}
\vspace{1cm}

\begin{equation}    6.25 = \end{equation}
\vspace{1cm}

\begin{equation}    2.01583 = \end{equation}
\vspace{1cm}

\end{multicols}

\vspace{1cm}

\section{N\'umeros decimales}

Resuelve las siguientes operaciones con decimales. Haz el acomodo correspondiente para resolver cada operaci\'on.
\setcounter{equation}{0}

\begin{multicols}{2}

\begin{equation}    123 + 0.13854    =\end{equation}\\
\vspace{2.5cm}

\begin{equation} 48  \div    1.2     =\end{equation}\\
\vspace{2.5cm}

\begin{equation}    12345.9457 - 235.78439  =\end{equation}\\
\vspace{2.5cm}

\begin{equation} 0.13    \div    15  =\end{equation}\\
\vspace{2.5cm}

\end{multicols}

\begin{multicols}{2}

\begin{equation} 678.2   \times  1.4 =\end{equation}\\
\vspace{2.5cm}

\begin{equation}    40 - 0.12    =\end{equation}\\
\vspace{2.5cm}

\begin{equation}    \sqrt{121} =\end{equation}\\
\vspace{2.5cm}

\begin{equation}    5^{3} =\end{equation}\\
\vspace{2.5cm}

\end{multicols}

\section{N\'umeros con signo}
Resuelve las siguientes operaciones de n\'umeros con signo.
\setcounter{equation}{0}

\begin{multicols}{2}

\begin{equation}    3 - 5 - 8 - 2 - 9 - 15 - 4 = \end{equation}
\vspace{2.5cm}

\begin{equation}    -10 + 20 -15 + 8 = \end{equation}
\vspace{2.5cm}

\begin{equation}    3 - (8+6) + (7-2) = \end{equation}
\vspace{2.5cm}

\begin{equation}    -\frac{1}{2} + \frac{6}{3} -2 + \frac{4}{8} = \end{equation}
\vspace{2.5cm}

\end{multicols}

\vspace{2.5cm}

\section{Geometr\'ia}

En el segmento de recta que se muestra a continuaci\'on coloca en el extremo izquierdo el n\'umero -5 y en extremo derecho el n\'umero 5. En \'este mismo  ubica las siguientes cantidades:


\begin{tabular}{p{2.5cm} p{2.5cm} p{2.5cm} p{2.5cm} p{2.5cm} p{2.5cm}}
(a) -1 &
(b) 3.1416 &
(c) 0 &
(d) 3.9 &
(e) 2 &
(f) -$\frac{5}{3} $
\end{tabular}

\vspace{2.4cm}

\begin{center}
    % \line(1,0){400}
    \rule[-1.5mm]{0.7pt}{3mm}\rule{5cm}{.7pt}\rule[-1.5mm]{0.7pt}{3mm}\rule{5cm}{.7pt}\rule[-1.5mm]{0.7pt}{3mm}
\end{center}

\vspace{1cm}

\newpage
Traza la mediatriz y la bisectriz, seg\'un sea el caso, para las siguientes rectas y \'angulos.

\begin{figure}[h!]
\centering
      \includegraphics[width=0.9\linewidth]{./med_bis}
\end{figure}

\vspace{-7cm}

Calcula el \'area y el per\'imetro de un hex\'agono regular asumiendo que la longitud de sus lados es $\ell = 4$\,cm y y la longitud de su apotema es $a = 3.5$\,cm.

\vspace{5cm}


Escribe la f\'ormula para calcular el \'area de las figuras que se mencionan a continuaci\'on:

\hspace{1cm} tri\'angulo \hfill cuadrado \hfill rect\'angulo \hspace{1cm}
\vspace{3cm}

Escribe con tus palabras el concepto de per\'imetro y \'area.
\vspace{3cm}


% \section{Notaci\'on cient\'ifica}

% A continuaci\'on se muestran una serie de cantidades. Convierte las que se encuentran en notaci\'on decimal a cient\'ifica y las que est\'an en notaci\'on cient\'ifica a decimal.
% \setcounter{equation}{0}


% \begin{multicols}{2}

% \begin{equation}    456.78 =    \end{equation}\\

% \begin{equation}    28000000 =    \end{equation}\\

% \begin{equation}    0.0000003924 =    \end{equation}\\

% \begin{equation}    2.49 \times 10 ^{5} =    \end{equation}\\

% \end{multicols}
% \begin{multicols}{2}

% \begin{equation}    3.1416 \times 10 ^{-3} =    \end{equation}\\

% \begin{equation}    7.1397492 \times 10^{4} =    \end{equation}\\

% \end{multicols}



\section{Ecuaciones}

Resuelve las ecuaciones que se presentan a continuaci\'on
\setcounter{equation}{0}

\begin{multicols}{2}

\begin{equation}    2x = 10     \end{equation}\\
\vspace{1.5cm}

\begin{equation}    5y + 7 = 12     \end{equation}\\
% \vspace{2.5cm}

\begin{equation}    5w - 20 = 15     \end{equation}\\
\vspace{1.5cm}

\begin{equation}    2x -6 + 4x = 18     \end{equation}\\
% \vspace{2.5cm}

\end{multicols}
% \vspace{1cm}

% \newpage
\section{Problemas}

A continuaci\'on se presentan una serie de problemas de distinto tipo en los que se requiere utilizar operaciones b\'asicas o proporcionalidad. Resuelve cada uno de ellos.\\
\setcounter{equation}{0}


\stepcounter{equation}
\noindent \theequation \hspace{4mm}
El viejo McDonald ten\'ia una granja. En esa granja se pueden alimentar por 8 d\'ias a 12 vacas con cierta cantidad de pastura. ?`Cu\'anto durar\'a la misma cantidad de pastura si se agregan 5 vacas nuevas al ganado?

\vspace{5cm}

\stepcounter{equation}
\noindent \theequation \hspace{4mm}
Buddy Valastro, famoso repostero de N.J. tiene una pasteler\'ia de renombre mundial famosa por sus pasteles esculturales. Para preparar el relleno de  18 \emph{cannolis} necesita 1.5\,Kg de queso \emph{ricotta} fresco y 150\,g de cerezas confitadas, entre otros ingredientes. Si se desean preparar s\'olo 10 \emph{cannolis}, ?`cu\'anto se necesitar\'a de los mismos ingredientes?

\vspace{5cm}

\stepcounter{equation}
\noindent \theequation \hspace{4mm}
La comercial Mexicana, debido a sus promociones de Julio Regalado, tuvo en oferta los helados con un 25\% de descuento. El costo original de tres porciones individuales de helado H\"aagen-Dasz son \$104.70, ?`cu\'al fue la cantidad descontada? ?`Cuanto es el total que se pag\'o despu\'es del descuento?

\vspace{5cm}

\stepcounter{equation}
\noindent \theequation \hspace{4mm}
En cierta tienda de m\'usica en l\'inea se tienen que pagar \$180.00 para descargar 18 canciones. ?`Cu\'al es el costo de cada canci\'on? ?`Cu\'anto se tendr\'ia que pagar por 10 canciones?

\vspace{7cm}

\stepcounter{equation}
\noindent \theequation \hspace{4mm}
Para amueblar un sal\'on de clases se requiere comprar treinta mesa-bancos, un escritorio, una silla, dos libreros, una mesa para computadora, una computadora, un proyector y un pintarr\'on. Acorde a los precios unitarios que se muestran a continuaci\'on determina el costo total de mobiliario del sal\'on de clases:

\begin{tabular}{lr}
mesa-banco:&    \$900.00 \\
escritorio:&    \$750.00 \\
Silla:&         \$230.00 \\
librero:&       \$7000.00 \\
mesa para computadora:& \$1750.00\\
computadora:&   \$5380.00 \\
proyector:&     \$10,250.00 \\
pintarr\'on:&   \$5,200.00
\end{tabular}

\vspace{2cm}

\stepcounter{equation}
\noindent \theequation \hspace{4mm}
Un arquitecto tiene como proyecto el construir una fuente. Para calcular los gastos de material necesita conocer el per\'imetro y el \'area que tendr\'a la estructura. Calcula estos par\'ametros si la fuente a construir tiene forma circular y su radio es $r = 8$\,m.

\vspace{5cm}

\stepcounter{equation}
\noindent \theequation \hspace{4mm}
?`Cu\'al es el di\'ametro de la fuente que se menciona en el problema anterior?

\vspace{5cm}


\stepcounter{equation}
\noindent \theequation \hspace{4mm}
Explica con tus palabras la diferencia entre proporcionalidad directa e inversa.


\end{document}
