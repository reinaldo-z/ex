\newcommand{\seccion}{SECUNDARIA INCORPORADA A LA SEG }
\newcommand{\descripcion}{Examen Bimestral, Cuarto Bimestre}
\newcommand{\grado}{Primero de secundaria}
\newcommand{\ciclo}{Ciclo escolar: 2015--2016}
\newcommand{\papel}{legalpaper} %letter, legalpaper ...
\newcommand{\fecha}{5 de mayo de 2016}
\author{Ing. Arturo Canedo M\'endez \\ M. en C. Reinaldo Zapata}
% \author{M. en C. Reinaldo Zapata}

\documentclass[11pt]{article}
\usepackage[\papel]{geometry}

\title{\flushleft \seccion \\ \descripcion \\  \grado \\ \ciclo}

\newcommand\BackgroundLogo{
\put(162,365){
\parbox[b][\paperheight]{\paperwidth}{%
\vfill
\centering
\includegraphics[width=5cm,height=2.5cm,keepaspectratio]{/Users/reinaldo/Documents/clases/jassa/logo}%
\vfill
}}}

\hyphenation{con-ti-nua-ción}

\usepackage{enumitem}
\usepackage[T1]{fontenc} %fuentes
\usepackage{lmodern} %fuente mejorada
\usepackage[spanish]{babel}
\decimalpoint
\usepackage{fullpage}
\usepackage{multicol}
\usepackage{graphicx}
\usepackage{eso-pic}
\usepackage{multirow}
\usepackage{subfigure}
\usepackage{tikz}
\usetikzlibrary{shapes.geometric}



%Modificación del formato de las ecuaciones y el numerado de las mismas
\usepackage[leqno,fleqn]{amsmath}
\makeatletter
  \def\tagform@#1{\maketag@@@{#1\@@italiccorr}}
\makeatother
\renewcommand{\theequation}{\fbox{\textbf{\arabic{equation}}}}


\begin{document}
\AddToShipoutPicture*{\BackgroundLogo}
\ClearShipoutPicture
\date{\fecha}
\maketitle
% \thispagestyle{empty}



Nombre del alumno:\,\line(1,0){244}\,.\hspace*{.2cm} \hfill Aciertos:
$\dfrac{\qquad \qquad}{50}$.

\vspace{3mm}
\indent Primero de secundaria, grupo:\,\line(1,0){35}\,. No. de
lista:\,\line(1,0){35}\,.

\vspace{5mm}

El prop\'osito de todo examen es poner a prueba los conocimientos de cada alumno
para calificar as\'i su desempe\~no y aprendizaje. Contesta correctamente, en
cada secci\'on, tantos reactivos como te sea posible. Todas y cada una de las
operaciones deber\'as escribirlas en esta hoja donde se imprimi\'o el examen y
los resultados finales deber\'an estar escritos con tinta.

\section{N\'umeros con signo}

Una pareja decide ahorrar mensualmente para amueblar su nueva casa antes de
mudarse a ella. En julio ahorran \$10,000. En agosto ahorran \$15,000 pero
gastan \$6,000 en un refrigerador que estaba en descuento. Para septiembre no 
pudieron ahorrar pero gastaron \$4,500 en una lavadora. Para inicios de octubre 
ahorraron \$7,000 pero a mediados de mes compraron un comedor de \$10,000. En 
noviembre no ahorraron ni gastaron. Por \'ultimo, en diciembre juntaron sus
aguinaldos y ahorraron \$30,000. ?`Cu\'anto tienen ahorrado a fin de a\~no?
\hfill \textbf{(4aciertos)}


\vspace{4cm}
Resuelve las siguientes operaciones de n\'umeros con signo. 
\hfill \textbf{(8aciertos)}
\begin{multicols}{2}

\begin{equation}
-(3 +5) -6 +8 -6 =
\end{equation}

\vspace{3cm}
\begin{equation}
-20 +10 -(15 -9) =
\end{equation}

\vspace{3cm}
\begin{equation}
4.16 -2.5 -3.15 + 0.975 =
\end{equation}

\vspace{3cm}
\begin{equation}
- \frac{2}{5} + \frac{2}{3} + \frac{1}{15} - \frac{1}{2} =
\end{equation}

\vspace{3cm}

\end{multicols}

\newpage

\setcounter{equation}{0}
\section{Ecuaciones}

Resuelve las ecuaciones que se plantean a continuaci\'on. \hfill \textbf{(4 aciertos)}

\begin{multicols}{2}

\begin{equation}
4x + 10 = 2
\end{equation}

\vspace{3cm}
\begin{equation}
\frac{1}{2} x + \frac{3}{10} = \frac{7}{5}
\end{equation}

\vspace{3cm}

\end{multicols}

\vspace{3.5cm}
Resuelve planteando la ecuaci\'on correspondiente: ?`C\'ual es el n\'umero que 
al multiplicarlo por 5 y restarle 4 da como resultado 11?
\hfill \textbf{(4 aciertos)}

\vspace{3.5cm}

\section{Probabilidad, estad\'istica y proporcionalidad}
En m\'exico hay cerca de 125 millones de habitantes. Acorde al censo hecho por
INEGI se tiene que el 27\% de la poblaci\'on se encuentra en el rango de edad
infantil (0-14 a\~nos), 81.25 millones de habitantes son adolescentes o adultos
(15-64a\~nos) y el resto son adultos en plenitud (65 o m\'as a\~nos). Completa
la tabla que se muestra a continuaci\'on y traza su correspondiente gr\'afica
circular colocando los porcentajes sobre el c\'irculo.
\hfill \textbf{(10 aciertos)}

\begin{center}
{\Large
\begin{tabular}{|l|c|c|c|}

\hline
\hline
Rango de edad & No. de pobladores & Porcentaje & \'Angulo \\
\hline
\hline
Infantes & & 27\% &  \\
\hline
Adolescentes/adultos & 81.25 millones & &  \\
\hline
Adultos mayores & &  &  \\
\hline
\hline
Total & 125 millones & 100\% & 360$^{\circ}$ \\
\hline
\hline

\end{tabular}
}
\end{center}

\vspace{3mm}
\begin{tikzpicture}[scale=0.85]
    \draw[line width=0.2mm, black] (0,0) circle (3.9cm);
    \draw [line width=0.2mm, black ] (0,0) -- (0,3.9) node [right] {};;
    \draw [line width=0.5mm, black ] (0,-0.2) -- (0,0.2) node [right] {};;
    \draw [line width=0.5mm, black ] (-0.2,0) -- (0.2,0) node [right] {};;

    \draw[line width=0.2mm, black] (4.5,1.5)  circle (3mm) node [right] {\Large \hspace{6mm}Infantes};
    \draw[line width=0.2mm, black] (4.5,0.0)  circle (3mm) node [right] {\Large \hspace{6mm}Adolescentes/adultos};
    \draw[line width=0.2mm, black] (4.5,-1.5) circle (3mm) node [right] {\Large \hspace{6mm}Adultos mayores};
\end{tikzpicture}  

\newpage
\vspace{3mm}
El profesor de matem\'aticas meti\'o 7 pelotas amarillas, 10 rojas, 9 azules,
13 verdes y 11 naranjas dentro de una bolsa verde. Expresa de forma
fraccional, decimal y porcentual la probabilidad de sacar una pelota al azar de
cada color.
\hfill \textbf{(6 aciertos)}

\vspace{5cm}
Cierto fabricante de autos de colecci\'on hace r\'eplicas a escala 1:18 de
distintos autos. Si un Lamborghini veneno tiene una longitud de 5.02\,m y un
ancho de 2.07\,m, ?`cu\'ales ser\'an las medidas del modelo a escala? Si
adem\'eas se hace otro modelo tres veces m\'as grande que el anterior,
?`cu\'ales ser\'an sus dimensiones?
\hfill \textbf{(6 aciertos)}


\vspace{4cm}
\section{Geometr\'ia}
Haciendo uso de la circunferencia que se muestra a continuaci\'on haz el trazo
de un hex\'agono regular inscrito. Traza su apotema. 

Asumiendo que su longitud lateral
es $\ell=5.9$\,cm y la de su apotema $a=4$\,cm, calcula su per\'imetro y su
\'area. \hfill \textbf{(8 aciertos)}

\vspace{3mm}
\begin{tikzpicture}[scale=0.85]
    \draw[line width=0.2mm, black] (0,0) circle (5cm);
    \draw [line width=0.2mm, black ] (0,-0.2) -- (0,0.2) node [right] {};;
    \draw [line width=0.2mm, black ] (-0.2,0) -- (0.2,0) node [right] {};;
\end{tikzpicture}  


\end{document}






